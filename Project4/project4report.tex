\documentclass[12pt,a4paper]{article}
\usepackage[utf8]{inputenc}
\usepackage[margin=1in]{geometry}
\usepackage{graphicx}
\usepackage{amsmath}
\usepackage{amsfonts}
\usepackage{enumerate}
\usepackage{listings}
\usepackage{xcolor}
\usepackage{float}
\usepackage{booktabs}
\usepackage{calc}

\definecolor{codegreen}{rgb}{0,0.6,0}
\definecolor{codegray}{rgb}{0.4,0.4,0.4}
\definecolor{codepurple}{rgb}{0.58,0,0.82}
\definecolor{backcolour}{rgb}{0.95,0.95,0.92}

\lstdefinestyle{mystyle}{
    backgroundcolor=\color{backcolour},   
    commentstyle=\color{codegreen},
    keywordstyle=\color{blue},
    numberstyle=\tiny\color{codegray},
    stringstyle=\color{codepurple},
    basicstyle=\sffamily\footnotesize,
    breakatwhitespace=false,         
    breaklines=true,                 
    captionpos=b,                    
    keepspaces=true,                 
    numbers=left,                    
    numbersep=5pt,                  
    showspaces=false,                
    showstringspaces=false,
    showtabs=false,                  
    tabsize=4
}

\lstset{style=mystyle}

% Title and Author
\title{TMA4162 Computational algebra, Project 4}
\author{Andreas Moe}
\date{\today}

\begin{document}

\maketitle

\section*{Task 1, Random Squares Factoring}
\begin{itemize}
    \setlength{\itemsep}{2em}
    \item
    The random squares method requires factoring smaller numbers via trial division.
    This implementation returns a list of exponents corresponding to the factors in the factor base
    \lstinputlisting[language=Python, caption={Trial Division Factoring}, label={lst:trialdivfactoring}]
    {temp/func-random_squares-trial_division_factoring.py}
    \item
    Stage 1 of the algorithm generates relations of the form:
    \[x^2 \equiv \prod_{i=1}^s {p_i}^{e_i} \pmod{N}\]
    The code below return a vector of random x values and a corresponding matrix of e values.
    \lstinputlisting[language=Python, caption={Stage 1}, label={lst:randomsquaresstage1}]
    {temp/func-random_squares-random_squares_stage_1.py}
    \item
    The next step is to find a non-trivial linear dependence mod 2 for the exponents.
    The nullspace of the exponent matrix over \(\mathbb{F}_2\) is computed using the galois library.
    The first element \(\lambda\) of the nullspace is then returned along another vector fs.

    fs = \(\lambda\cdot exponents / 2\).
    These two vectors will be used to compute X and Y whose squares will be congruent mod N.
    \lstinputlisting[language=Python, caption={Non-trivial Linear Dependence}, label={lst:lindep}]
    {temp/func-random_squares-non_trivial_lin_dep.py}
    \item
    X and Y are computed as follows:
    \[X \equiv \prod_{j=1}^{t} {x_j}^{\lambda_j} \pmod{N}, \quad Y \equiv \prod_{i=1}^{s} {p_i}^{f_i} \pmod{N}\]
    where t = number of relations, s = size of factor base
    \lstinputlisting[language=Python, caption={Computing the squares}, label={lst:computesquares}]
    {temp/func-random_squares-compute_squares.py}
    \item
    The full algorithm uses the above functions and finally computes \(gcd(X-Y, N)\), which will hopefully be a
    non-trivial factor of N.
    \lstinputlisting[language=Python, caption={Full algorithm}, label={lst:randomsquaresfactoring}]
    {temp/func-random_squares-random_squares_factoring.py}
\end{itemize}

\section*{Task 2a, Quadratic Sieve}

The quadratic sieve is a faster way to find relations.
This implementation takes in a factor base, N and a sieve size.
The sieve is generated from square numbers mod N, close to zero.
It then uses the  sympy.ntheory.residue\_ntheory library to calculate the solutions to \(x^2 \equiv N \pmod{p}\) for
each p in the factor base.
These solutions are then used as the basis for the sieve process, where the sieve element at each multiple of p are
divided by p.
And the corresponding entries in the exponent matrix are incremented.
Unfortunately, this means that prime powers can't be factored, which severely limits the performance of this
implementation.

Finally, the algorithm filters out number that still have remaining factors.

\lstinputlisting[language=Python, caption={Quadratic Sieve}, label={lst:quadraticsieve}]
{temp/func-quadratic_sieve-quadratic_sieve.py}

This factoring algorithm uses the quadratic sieve and the functions from task 1 to factor an integer N.

\lstinputlisting[language=Python, caption={Quadratic Sieve Factoring}, label={lst:quadraticsievefactoring}]
{temp/func-quadratic_sieve-quadratic_sieve_factoring.py}

\section*{Task 4}

I tried to redo the factorisations from Brillhart and Selfridge\cite{brillhartselfridge}, but even the first example
was too large for my implementation.

\newpage
\begin{appendix}
\section*{Appendix}
    Code is available at https://github.com/andrmoe/ComputationalAlgebra
    \lstinputlisting[language=Python, caption={random\_squares.py}, label={lst:python_code1}]{random_squares.py}
    \lstinputlisting[language=Python, caption={quadratic\_sieve.py}, label={lst:python_code2}]{quadratic_sieve.py}
    \lstinputlisting[language=Python, caption={examples\_brillhart\_and\_selfridge.py}, label={lst:python_code3}]{examples_brillhart_and_selfridge.py}
    \lstinputlisting[language=Python, caption={experiment.py}, label={lst:python_code4}]{experiment.py}
    \lstinputlisting[language=Python, caption={test.py}, label={lst:python_code5}]{test.py}
\end{appendix}


\begin{thebibliography}{1}

\bibitem{brillhartselfridge}
John Brillhart, J. L. Selfridge, ``Some factorizations of 2n ±1 and related results,'' \textit{Mathematics of Computation}, vol. 21, no. 97, pp. 87--96, 1967.

\end{thebibliography}


\end{document}
