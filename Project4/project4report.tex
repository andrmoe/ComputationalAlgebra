\documentclass[12pt,a4paper]{article}
\usepackage[utf8]{inputenc}
\usepackage[margin=1in]{geometry}
\usepackage{graphicx}
\usepackage{amsmath}
\usepackage{amsfonts}
\usepackage{enumerate}
\usepackage{listings}
\usepackage{xcolor}
\usepackage{float}
\usepackage{booktabs}
\usepackage{calc}

\definecolor{codegreen}{rgb}{0,0.6,0}
\definecolor{codegray}{rgb}{0.4,0.4,0.4}
\definecolor{codepurple}{rgb}{0.58,0,0.82}
\definecolor{backcolour}{rgb}{0.95,0.95,0.92}

\lstdefinestyle{mystyle}{
    backgroundcolor=\color{backcolour},   
    commentstyle=\color{codegreen},
    keywordstyle=\color{blue},
    numberstyle=\tiny\color{codegray},
    stringstyle=\color{codepurple},
    basicstyle=\sffamily\footnotesize,
    breakatwhitespace=false,         
    breaklines=true,                 
    captionpos=b,                    
    keepspaces=true,                 
    numbers=left,                    
    numbersep=5pt,                  
    showspaces=false,                
    showstringspaces=false,
    showtabs=false,                  
    tabsize=4
}

\lstset{style=mystyle}

% Title and Author
\title{TMA4162 Computational algebra, Project 4}
\author{Andreas Moe}
\date{\today}

\begin{document}

\maketitle

\section*{Task 1, Random Squares Factoring}
\begin{itemize}
    \setlength{\itemsep}{2em}
    \item
    The random squares method requires factoring smaller numbers via trial division.
    This implementation returns a list of exponents corresponding to the factors in the factor base
    \lstinputlisting[language=Python, caption={Trial Division Factoring}, label={lst:trialdivfactoring}]
    {temp/func-random_squares-trial_division_factoring.py}
    \item
    Stage 1 of the algorithm generates relations of the form:
    \[x^2 \equiv \prod_{i=1}^s {p_i}^{e_i} \pmod{N}\]
    The code below return a vector of random x values and a corresponding matrix of e values.
    \lstinputlisting[language=Python, caption={Stage 1}, label={lst:randomsquaresstage1}]
    {temp/func-random_squares-random_squares_stage_1.py}
    \item
    The next step is to find a non-trivial linear dependence mod 2 for the exponents.
    The nullspace of the exponent matrix over \(\mathbb{F}_2\) is computed using the galois library.
    The first element \(\lambda\) of the nullspace is then returned along another vector fs.

    fs = \(\lambda\cdot exponents / 2\).
    These two vectors will be used to compute X and Y whose squares will be congruent mod N.
    \lstinputlisting[language=Python, caption={Non-trivial Linear Dependence}, label={lst:lindep}]
    {temp/func-random_squares-non_trivial_lin_dep.py}
    \item
    X and Y are computed as follows:
    \[X \equiv \prod_{j=1}^{t} {x_j}^{\lambda_j} \pmod{N}, \quad Y \equiv \prod_{i=1}^{s} {p_i}^{f_i} \pmod{N}\]
    where t = number of relations, s = size of factor base
    \lstinputlisting[language=Python, caption={Computing the squares}, label={lst:computesquares}]
    {temp/func-random_squares-compute_squares.py}
    \item
    The full algorithm uses the above functions and finally computes \(gcd(X-Y, N)\), which will hopefully be a
    non-trivial factor of N.
    \lstinputlisting[language=Python, caption={Full algorithm}, label={lst:randomsquaresfactoring}]
    {temp/func-random_squares-random_squares_factoring.py}
\end{itemize}



\newpage
\begin{appendix}
\section*{Appendix}
    Code is available at https://github.com/andrmoe/ComputationalAlgebra
    \lstinputlisting[language=Python, caption={test.py}, label={lst:python_code1}]{random_squares.py}
    \lstinputlisting[language=Python, caption={test.py}, label={lst:python_code8}]{test.py}
\end{appendix}

\end{document}
