\documentclass[12pt,a4paper]{article}
\usepackage[utf8]{inputenc}
\usepackage[margin=1in]{geometry}
\usepackage{graphicx}
\usepackage{amsmath}
\usepackage{amsfonts}
\usepackage{enumerate}
\usepackage{listings}
\usepackage{xcolor}
\usepackage{float}
\usepackage{booktabs}
\usepackage{calc}

\definecolor{codegreen}{rgb}{0,0.6,0}
\definecolor{codegray}{rgb}{0.4,0.4,0.4}
\definecolor{codepurple}{rgb}{0.58,0,0.82}
\definecolor{backcolour}{rgb}{0.95,0.95,0.92}

\lstdefinestyle{mystyle}{
    backgroundcolor=\color{backcolour},   
    commentstyle=\color{codegreen},
    keywordstyle=\color{blue},
    numberstyle=\tiny\color{codegray},
    stringstyle=\color{codepurple},
    basicstyle=\sffamily\footnotesize,
    breakatwhitespace=false,         
    breaklines=true,                 
    captionpos=b,                    
    keepspaces=true,                 
    numbers=left,                    
    numbersep=5pt,                  
    showspaces=false,                
    showstringspaces=false,
    showtabs=false,                  
    tabsize=4
}

\lstset{style=mystyle}

% Title and Author
\title{TMA4162 Computational algebra, Project 2}
\author{Andreas Moe}
\date{\today}

\begin{document}

\maketitle

\section*{Task 1, Theory}
Define \(S_n = \{2^n, 2^n+1, \dots , 2^{n+1}-1\}\)
\begin{enumerate}[a)]
    \item
    Let \(X\) be a uniform random variable over \(S_n\).
    \begin{gather*}
        Let\ p = P(X\ is\ a\ prime) = \frac{number\ of\ primes\ in\ S_n}{|S_n|}
        = \frac{\pi(2^{n+1}-1) - \pi(2^n-1)}{2^n}\\
        \approx 2^{-n}\left(\frac{2^{n+1}}{log(2^{n+1})}-\frac{2^n}{log(2^n)}\right)
        = \frac{2}{(n+1)\ log\ 2}-\frac{1}{n\ log\ 2}
        = \frac{2n - (n+1)}{(n+1)n\ log\ 2}
        \approx \frac{1}{n\ log 2}\\
    \end{gather*}

    \vspace{20pt}
    We are interested in the expected number of candidates that must be tested before a prime is found. The number of failed candidates follows a negative binomial distribution with one success (r=1). The expected number of trials to see \(r\) successes in a negative binomial distribution is \(\frac{r}{p} = \frac{1}{p}\) in our case.
    The expected number of trials is therefore \(\approx \mathbf{n\ \log\ 2}\).
    The table below shows this approximation for several values of \(n\).
        {\begin{table}[H]
             \centering
             \caption{Expected Number of Prime Candidates}
             \label{tab:exp_prime_candidates}
             \begin{tabular}{rr}
\toprule
n & Candidates \\
\midrule
10 & 6.931472 \\
50 & 34.657359 \\
100 & 69.314718 \\
500 & 346.573590 \\
1000 & 693.147181 \\
\bottomrule
\end{tabular}

        \end{table}}

    \item
    Let \(A = \{a, a+1, \dots , a + d - 1\}\) where \(a\) is selected at random from \(S_n\).
    We will assume that \(A \subset S_n\).

    We want to estimate the probability that \(A\) contains at least one prime.

    \begin{gather*}
        P(A\ contains\ at\ least\ one\ prime) = 1 - P(A\ contains\ no\ primes)\\
        P(A\ contains\ no\ primes) \approx (1-p)^d = \left(1-\frac{1}{n\ log\ 2}\right)^d
    \end{gather*}

    We'll demand that \(P(A\ contains\ no\ primes) < q = 0.01\)
    \begin{gather*}
        \left(1-\frac{1}{n\ log\ 2}\right)^d < q \\
        d \cdot log\left(1-\frac{1}{n\ log\ 2}\right) < log\ q \\
        d > \frac{log\ q}{log\left(1-\frac{1}{n\ log\ 2}\right)} =
        \frac{log\ q}{log\left(\frac{(n\ log\ 2) - 1}{n\ log\ 2}\right)} =
        \frac{log\ q}{log\left(\frac{n - 1/log(2)}{n}\right)} =
        \frac{log\ q}{log(n - \frac{1}{log\ 2}) - log(n))} \\
        \approx \frac{log\ q}{\left(-\frac{1}{log\ 2}\right) \cdot log'(n)} =
        \frac{- log\ q}{\frac{1}{log\ 2} \cdot \frac{1}{n}} =
        n \cdot log\ 2 \cdot log\left( \frac{1}{q} \right)
    \end{gather*}
    Setting \(q = 0.01\) then gives us that \(\mathbf{d > 3.19n}\).
    The table below shows lower bounds on d for several values of \(n\).
        {\begin{table}[H]
             \centering
             \caption{Lower bounds on range size}
             \label{tab:lower_bounds_for_d}
             \input{temp/range_size_table}
        \end{table}}

    The probability of picking a range with at least one prime, if we use the suggested range size \(d\) from above,
    is over \(\mathbf{99\%}\).

    In the case of a), every prime number is equally likely to be picked.
    The distribution of the prime numbers that are found is therefore identical to the distribution of primes in
    \(S_n\).
    And the prime number theorem tells us that there are more primes in the first half of \(S_n\) than the second.

    The primes that are found in b), however, are very close to \(a\).
    Which is a uniformly distributed random variable.
    The primes that are found, will therefore also be uniformly distributed.

    \item
    Let \(A\) be the same as before. And define \(A_i\) to be what remains after filtering out numbers that are
    divisible by the \(i\) smallest primes. Where the \(i\) smallest primes are denoted \(p_1, p_2, \dots , p_i\).
    \begin{gather*}
        |A_i| \approx \left( 1 - \frac{1}{p_i} \right)\\
        |A_i| \approx \prod_{j=1}^{i}\left( 1 - \frac{1}{p_i} \right)|A| =
    \prod_{j=1}^{i}\left( 1 - \frac{1}{p_i} \right)d\\
    \end{gather*}
    The number of primes in \(A_i\) is the same as in \(A\).
    The probability that an element in \(A_i\) is prime is therefore:
    \begin{gather*}
        p_i = \frac{number\ of\ primes\ in\ A}{|A_i|} \approx \frac{pd}{\prod_{j=1}^{i}\left( 1 - \frac{1}{p_i} \right)d}\\
        = \frac{1}{(n\ \log\ 2)\prod_{j=1}^{i}\left( 1 - \frac{1}{p_i} \right)}\\
    \end{gather*}
    We can substitute \(n \Longrightarrow n\prod_{j=1}^{i}\left( 1 - \frac{1}{p_i} \right)\) in c) to obtain a value
    for d, where we'll be 99\% sure that we will find a prime.
    \[d_i > 3.19n\prod_{j=1}^{i}\left( 1 - \frac{1}{p_i} \right)\]
    The table below shows lower bounds on d for several values of \(n\). \(d_0 = d\)
        {\begin{table}[H]
             \centering
             \caption{Lower bounds on range size, sieved}
             \label{tab:sievedlowerbounds}
             \input{temp/sieve_table}
        \end{table}}
    \end{enumerate}

\section*{Task 2, Practice}

\begin{enumerate}[a)]
    \setlength{\itemsep}{10pt}
    \item
    The Fermat test works by selecting a random number and checking if Fermat's little theorem holds.
    The algorithm returns composite (False) if it doesn't hold, then repeats this 100 times.
    If the theorem holds for all 100 tests, the algorithm concludes that \(p\) is probably prime and returns True.
    \vspace{20pt}
    \lstinputlisting[language=Python, caption={Fermat test}, label={lst:fermattest}]{temp/fermat_test.py}
    \item
    A generic prime finding algorithm takes a stream of prime candidates and a prime test.
    It tests each candidate, and returns the first prime it finds.
    \lstinputlisting[language=Python, caption={Generic prime test}, label={lst:findprimegeneric}]{temp/find_prime_generic.py}
    \vspace{20pt}
    This algorithm samples candidates from \(S_n\) as a stream of candidates.
    \lstinputlisting[language=Python, caption={Basic prime finder}, label={lst:findprime}]{temp/find_prime.py}
    \item
    This prime test works by iterating through a list of precomputed small primes.
    The list is part of the file precomputed\_numbers.py and consists of all primes, smaller than 10000.
    If the candidate is divisible by a precomputed prime (and it's not equal to the prime), it returns composite.
    If the candidate is not divisible by any of the precomputed primes, a Fermat test is run to make a final decision.
    \lstinputlisting[language=Python, caption={Trial division test}, label={lst:trialdivisiontest}]{temp/trial_division_test.py}
    \item
    The prime sieve constructs a list of prime candidates, by starting with a list of consecutive integers from
    \(n_{min}\) to \(n_{max}\) (exclusive).
    Then it removes all numbers that are divisible by a small precomputed prime \(p\).
    It starts by finding the smallest number \(m\) in the range that is divisible \(p\), then skips \(p\) indices ahead
    again and again until it reaches the end of the range.
    If \(m = p\), then \(m\) is not removed, since it's prime, but all subsequent multiples are removed.
    This process is repeated for as many small primes as specified.

    \(m\) is computed like this:
    \[m = \left\lfloor \frac{n_{min}-1}{p} + 1 \right\rfloor \cdot p\]
    \lstinputlisting[language=Python, caption={Sieve prime finder}, label={lst:sieve}]{temp/prime_sieve.py}
    To see why the formula for \(m\) works, let \(c\) be such that:
    \[m = \left( \frac{n_{min}-1}{p} + 1 - c \right) \cdot p\]
    It is clear that \(c \in [0, 1)\).
    \[m = n_{min} - 1 + p - cp > n_{min} - 1 + p - p = n_{min} - 1 \Rightarrow m \geq n_{min}\]
    Also:
    \[m - p = n_{min} - 1 - cp \leq n_{min} - 1 \Rightarrow m - p < n_{min}\]
    This shows that \(m\) is the smallest multiple of \(p\), that is greater than \(n_{min}\).

    \item


\end{enumerate}

\newpage
\begin{appendix}
\section*{Appendix}
    Code is available at https://github.com/andrmoe/ComputationalAlgebra
    \lstinputlisting[language=Python, caption={prime.py}, label={lst:python_code1}]{prime.py}
    \lstinputlisting[language=Python, caption={test.py}, label={lst:python_code2}]{test.py}
    \lstinputlisting[language=Python, caption={experiment.py}, label={lst:python_code3}]{experiment.py}
\end{appendix}

\end{document}
